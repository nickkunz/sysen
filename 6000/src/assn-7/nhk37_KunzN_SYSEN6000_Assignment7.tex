\documentclass{article}

%% doc settings
\hyphenchar\font=-1 % suppress hyphenation
\setlength\parindent{0pt} % suppress indentation
\usepackage[margin=1.25truein]{geometry} % set page margins

%% libraries 
\usepackage{array}
\usepackage{listings}
\usepackage{fancyhdr}
\usepackage{lastpage}
\usepackage{url}
\usepackage{xcolor}
\usepackage{hyperref}
\usepackage{amssymb}
\usepackage{amsthm}
\usepackage{amsmath}
\usepackage{natbib}
\usepackage{tikz}
\usepackage{pgfplots}
\usepackage{textcomp}
\usepackage{subcaption}

%% link viz
\hypersetup{
    colorlinks = true,
    linkcolor = red,
    urlcolor = red,
    citecolor = black
}

%% figs
\usetikzlibrary{
    shapes,
    arrows,
    positioning,
    calc
}

%% page nums
\pagestyle{fancy}
\fancyhf{}
\fancyfoot[C]{Pg. \thepage \space of \pageref*{LastPage}}
\renewcommand{\headrulewidth}{0pt}

%% begin doc
\begin{document}
\title{SYSEN 6000: Foundations of Complex Systems\\~\\
    \Large Decision Making Under Uncertainty
}
\author{
    Nick Kunz [NetID: \url{nhk37}] \hyperlink{nhk37@cornell.edu}{nhk37@cornell.edu}}
\date{November 11, 2022}
\maketitle
\thispagestyle{fancy}

%% body doc
\section*{Two-Stage Stochastic Mixed-Integer Linear Programming}
A company is considering to produce a product $C$, which can be manufactured with either Process II or Process III, both of which use raw material $B$. $B$ can be purchased from another company, or manufactured with Process I, which uses raw material $A$.\\

Process II and Process III are exclusive (at most one of them can be built). There are five possible outcomes of demand and sale price combinations for product $C$. The probability of each scenario corresponding to the demand and sale price, as well as all supplemental data are below:\\

\begin{center}
\begin{tabular}{ | m{52pt} | m{112pt}| m{96pt} | m{80pt} |}
    \hline
    \space & \textbf{Demand of $C$ (M tons)} & \centering \textbf{Sale Price (\$/ton)} & \centering \textbf{Probability} & 
    \hline
    \textbf{Scenario 1} & \centering 5 & \centering 2,000 & \centering 0.10 & 
    \hline
    \textbf{Scenario 2} & \centering 8 & \centering 1,900 & \centering 0.20 &
    \hline
    \textbf{Scenario 3} & \centering 10 & \centering 1,800 & \centering 0.40 &
    \hline
    \textbf{Scenario 4} & \centering 12 & \centering 1,600 & \centering 0.20 &
    \hline
    \textbf{Scenario 5} & \centering 15 & \centering 1,000 & \centering 0.10 &
    \hline
\end{tabular}
\end{center}

\begin{center}
\begin{tabular}{ | m{52pt} | m{96pt}| m{96pt}| m{96pt}| }
    \hline
    \space & \centering \textbf{Fixed CapEx (\$M)} & \centering \textbf{Var CapEx (\$/ton)} & \centering \textbf{Var OpEx (\$/ton)} &
    \hline
    \textbf{Phase I} & \centering 100 & \centering 250 & \centering 100 &
    \hline
    \textbf{Phase II} & \centering 150 & \centering 400 & \centering 150 &
    \hline
    \textbf{Phase III} & \centering 200 & \centering 550 & \centering 200 &
    \hline
\end{tabular}
\end{center}

\\~\\
The price for raw material $A$ is 250 (\$/ton) and the price for raw material $B$ is 450 (\$/ton).\\

The conversion rate for Process I is 0.90 of $A$ to $B$, Process II is 0.82 of $B$ to $C$, and Process III is 0.95 of $B$ to $C$.\\

The maximum supply of $A$ is 16 (M tons).

\break
Given the problem, maximizing expected profit can be formulated as:
\begin{equation}
\begin{split}
    \text{max \space} 
    & = -100x_1 - 250z_1 - 150x_2 - 400z_2 - 200x_3 - 550z_3 + \\
    & \sum_{s} p_s((y_s_3 + y_s_4) \cdot Price_s - 100y_s_1 - 250_y_s_1 / 0.90 - 450y_s_z - 150y_s_3 - 200y_s_4)\\
    \text{s.t.\space} 
    & \space y_s_1 \leq z_1\\
    & \space z_1 / 0.90 \leq 16 \cdot x_s_1\\
    & \space y_s_1 + y_s_2 = (y_s_3 / 0.82) + (y_s_4 / 0.95)\\
    & \space y_s_3 + y_s_4 \leq Demand_s\\
    & \space y_s_3 \leq z_2\\
    & \space z_2 \leq M \cdot x_2\\
    & \space y_s_4 \leq z_3\\
    & \space z_3 \leq M \cdot x_3\\
    & \space x_2 + x_3 = 1
\end{split}
\end{equation}

where the first stage decision variables are:
\begin{equation}
\begin{split}
    x_1 & = \text{Process I } [0,1] \text{ with capacity } z_1\\
    x_2 & = \text{Process II } [0,1] \text{ with capacity } z_2\\
    x_3 & = \text{Process III } [0,1] \text{ with capacity } z_3
\end{split}
\end{equation}

and the second stage decision variables are:
\begin{equation}
\begin{split}
    y_s_1 & = B \text{ manufactured in Process I}\\
    y_s_2 & = B \text{ purchased from another company}\\
    y_s_3 & = C \text{ manufactured in Process II}\\
    y_s_4 & = C \text{ manufactured in Process III}
\end{split}
\end{equation}

that when solved, the optimal value is \textbf{\$4,931.098M}, where:
    \begin{equation}\nonumber
        x_1 = 0, x_2 = 1, x_3 = 0\\
    \end{equation}
    \begin{equation}
        z_2 = 10\\
    \end{equation}

Considering this solution, it would be prudent to manufacture product $C$ with Process II. The corresponding capacity would be 10M tons.\\

It would also be prudent to obtain the raw material $B$ by purchasing it from another company, rather than manufacturing it with Process I with raw material $A$, where product $C$ should be sold in the following amounts for each scenario:\\

\begin{center}
\begin{tabular}{ | m{52pt} | m{92pt}| }
    \hline
    \textbf{} & \textbf{Sale of $C$ (M tons)} & 
    \hline
    \textbf{Scenario 1} & \centering 5 & 
    \hline
    \textbf{Scenario 2} & \centering 8 & 
    \hline
    \textbf{Scenario 3} & \centering 10 & 
    \hline
    \textbf{Scenario 4} & \centering 10 & 
    \hline
    \textbf{Scenario 5} & \centering 10 & 
    \hline
\end{tabular}
\end{center}

\break
In this case, the \textit{``value of perfect information''} can be reformulated as:
\begin{equation}
\begin{split}
    \text{max \space} 
    = & \sum_{s} p_s (-100x_s_1 - 250z_s_1 - 150x_s_2 - 400z_s_2 - 200x_s_3 - 550z_s_3) + \\
    & \sum_{s} p_s ((y_s_3 + y_s_4) \cdot Price_s - 100y_s_1 - (250_y_s_1 / 0.90) - 450y_s_z - 150y_s_3 - 200y_s_4)\\
    \text{s.t. \space} 
    & \space y_s_1 \leq z_s_1\\
    & \space z_s_1 / 0.90 \leq 16 \cdot x_s_1\\
    & \space y_s_1 + y_s_2 = (y_s_3 / 0.82) + (y_s_4 / 0.95)\\
    & \space y_s_3 + y_s_4 \leq Demand_s\\
    & \space y_s_3 \leq z_s_2\\
    & \space z_s_2 \leq M \cdot x_s_2\\
    & \space y_s_4 \leq z_s_3\\
    & \space z_s_3 \leq M \cdot x_s_3\\
    & \space x_s_2 + x_s_3 = 1
\end{split}
\end{equation}

where there are no first stage decision variables, and the second stage decision variables are:
\begin{equation}
\begin{split}
    x_s_1 & = \text{Process I } [0,1] \text{ with capacity } z_s_1\\
    x_s_2 & = \text{Process II } [0,1] \text{ with capacity } z_s_2\\
    x_s_3 & = \text{Process III } [0,1] \text{ with capacity } z_s_3\\
    y_s_1 & = B \text{ manufactured in Process I}\\
    y_s_2 & = B \text{ purchased from another company}\\
    y_s_3 & = C \text{ manufactured in Process II}\\
    y_s_4 & = C \text{ manufactured in Process III}
\end{split}
\end{equation}

that when solved, the optimal value is \$5,590.366M, so that:
\begin{equation}
    \$5,590.366 \text{M} - \$4,931.098 \text{M} = \$659.268\text{M}
\end{equation}

Hence, the \textit{``value of perfect information''} is \textbf{\$659.268M}.\\
\\~\\

\section*{Robust Two-Stage Stochastic Linear Programming}

The following is a derivation of the robust counterpart, given that:
\begin{equation}
\begin{split}
    \text{max \space} 
    & 10x_1 + 5x_2\\
    \text{s.t.\space}
    & \space (6 + u_1) x_1 + (2 + u_2) x_2 \leq 80, \, \forall (u_1, u_2) \in U
\end{split}
\end{equation}\\

where $U$ is the box uncertainty set:
\begin{equation}
    U = \{(u_1, u_2): |u_1| \leq 1, |u_2| \leq 1\}
\end{equation}

\break
which can begin with the expression:
\begin{equation}
\begin{split}
    \text{max \space} 
    & 10x_1 + 5x_2\\
    \text{s.t.\space}
    & \space 6x_1 + 2x_2 + \sqrt{(x_1)^2 + (x_2)^2} \leq 80
\end{split}
\end{equation}\\

and the box uncertainty set $U$, which can be expressed as a general polyhedral set:
\begin{equation}
    U = 
    \left\{ 
    (u_1, u_2) \Bigg|
        \begin{array}{lr}
            & u_1 \leq 1\\
            & u_2 \leq 1\\
            & -u_1 \leq 1\\
            & -u_2 \leq 1\\
        \end{array}
    \right\} = \{ (u_1, u_2) | \textbf{Wu} \leq \textbf{v} \}
\end{equation}

where:
\begin{equation}
    \textbf{W} = 
    \begin{bmatrix}
        1 & 0 & -1 & 0\\
        0 & 1 & 0 & -1
    \end{bmatrix}^T, \, \textbf{v} =  
    \begin{bmatrix}
        1 & 1 & 1 & 1
    \end{bmatrix}^T
\end{equation}

such that:
\begin{equation}
    \textbf{P} = \begin{bmatrix}
        1 & 0\\
        0 & 1
    \end{bmatrix}, \, \textbf{p} =  
    \begin{bmatrix}
        6\\
        2
    \end{bmatrix}, \, \textbf{q} = 
    \begin{bmatrix}
        0\\
        0
    \end{bmatrix}, \, r = 80
\end{equation}\\

This allows for the constraint $(6 + u_1) x_1 + (2 + u_2) x_2 \leq 80, \, \forall (u_1, u_2) \in U$ to be reformulated as:
\begin{equation}
\begin{split}
    & \lambda_1 + \lambda_2 + \lambda_3 + \lambda_4 \leq -6x_1 - 2x_2 + 80\\
    & \lambda_1 - \lambda_3 = x_1\\
    & \lambda_2 - \lambda_4 = x_2\\
    & \lambda_1, \lambda_2, \lambda_3, \lambda_4 \geq 0
\end{split} 
\end{equation}

such that $\lambda_1,\ldots, \lambda_4$ are dual variables corresponding to $\textbf{Wu} \leq \textbf{v}$.\\

\\~\\
Hence, the robust counterpart can be expressed as:
\begin{equation}
\begin{split}
    \text{max \space} 
    & 10x_1 + 5x_2\\
    \text{s.t. \space} 
    & \lambda_1 + \lambda_2 + \lambda_3 + \lambda_4 \leq -6x_1 - 2x_2 + 80\\
    & \lambda_1 - \lambda_3 = x_1\\
    & \lambda_2 - \lambda_4 = x_2\\
    & \lambda_1, \lambda_2, \lambda_3, \lambda_4 \geq 0
\end{split}
\end{equation}

% That when solved, the optimal value is $\cdot$, where:
% \begin{equation}\nonumber
%     \lambda_1  = \cdot, \lambda_2 = \cdot, \lambda_3 = \cdot, \lambda_4 = \cdot
% \end{equation}
% \begin{equation}
%     x_1 = , x_2 = 
% \end{equation}

\end{document}