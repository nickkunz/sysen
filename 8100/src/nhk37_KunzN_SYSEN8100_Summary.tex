\documentclass[11pt]{article}
\pagenumbering{gobble} % suppress page number
\hyphenchar\font=-1 % suppress hyphenation
\setlength\parindent{0pt} % suppress indentation

\usepackage[margin=1truein]{geometry}
\usepackage{url}
\usepackage{xcolor}
\usepackage{hyperref}

\hypersetup{
    colorlinks = true,
    linkcolor = red,
    urlcolor = red
}

\begin{document}
\title{SYSEN 8100-2: Systems Seminar Series\\~\\
    \Large Systems Summary
}
\author{
    Nick Kunz [NetID: \url{nhk37}] \hyperlink{nhk37@cornell.edu}{nhk37@cornell.edu}
}
\date{May 4, 2023}
\maketitle
Throughout the Spring 2023 semester, 10 presentations were given as a part of the Cornell Systems Seminar Series. These presentations discussed a wide range of topics related to systems engineering. They broadly emphasized the importance of sustainability, efficient resource management, and the environment. These lectures ranged from environmental engineering to electricity markets, agricultural technology, and smart buildings. This work pointed to the importance of interdisciplinary research in an effort to address global challenges such as climate change, sustainable agriculture, public health, urbanization, and energy.\\

One highlight of the lecture series was when Dr. Michael Charles, Provost’s New Faculty Fellow and Assistant Professor with the Department of Biological and Environmental Engineering at Cornell University, addressed the need for expanding system boundaries in sustainable design research to include socio-technical systems that integrate community-based input with computational modeling. His work focused on developing optimization frameworks along with community identified interests and applications. Prof. Madhur Srivastava, Assistant Research Professor in the Department of Chemistry and Chemical Biology at Cornell University, also addressed several adjacent issues related to community-focused research through personalized electricity consumption profiles to help manage load and reduce cost.\\

Another highlight was the discussion regarding the potential of controlled environment agriculture (CEA) by Prof. Murat Kacira, Director of the Controlled Environment Agriculture Center and Professor in the Department of Biosystems Engineering at the University of Arizona. He demonstrated the importance of co-optimizing environmental variables to improve resource efficiency and profitability. Prof. Kacira highlighted the potential of optimizing environmental variables and implementing wavelength shifting and semi-transparent materials, which could be used to improve crop yields. In addition, using sensors and IoT devices, combined with computer vision to monitor and improve greenhouse environment control, providing warnings for crop stress and growth status.\\

An additional theme that emerged from the work of Prof. Wang Shengwei, Director of Research Institute for Smart Energy at The Hong Kong Polytechnic University, was smart building technology for achieving carbon neutrality in high-density cities by enhancing energy flexibility and efficiency in high-rises. Also, Dr. Tianzhen Hong, Senior Scientist with the Building Technology and Urban Systems Division of the Lawrence Berkeley National Laboratory, addressed the application of energy modeling to identify and evaluate methods for improving energy efficiency and demand flexibility in buildings, as well as public health hazards of extreme weather events and power outages. \\

The topic of extreme weather events and power outages was also addressed by Dr. Hong Chen, Principal Engineer at PJM Interconnection, Vice President of IEEE PES Technical Activities, and Chair of IEEE PES Technical Council. Her lecture on wholesale electricity market operations addressed a range of topics in this regard, such as market uncertainty, gas-electric coordination, renewable energy, storage, and demand response. Prof. Mostafa Ardakani, Associate Professor in the Department of Electrical and Computer Engineering at the University of Utah, also focused on techniques to reduce power outages due to extreme weather events by applying preventative power system dispatch methods.\\

In another presentation related to energy, but with a different focus, was by Dr. Siming You, Senior Lecturer in the James Watt School of Engineering at the University of Glasgow, UK. His research emphasizes the role of sustainable bioenergy production and waste management in achieving Climate Action and Sustainable Cities and Communities. Dr. You applied machine learning techniques to more accurately assess environmental impacts related to municipal solid waste, food waste, cow slurry, and rapeseed oil. One interesting aspect of the presentation was the use of prediction systems to overcome the lack of reliable data related to accurate and flexible environmental impact assessment models.\\

Prof. Jinfeng Liu, Professor in the Department of Chemical and Materials Engineering at the University of Alberta, presented his research on closed-loop agricultural irrigation technology for water sustainability to address water and food scarcity, which focused on modeling and control methods. Finally, Prof. Jingzheng Ren, Associate Professor at The Hong Kong Polytechnic University, presented the development of various generic mathematical models for sustainability assessment, improvement, and optimization of energy and industrial systems. Prof. Ren demonstrated how these models can be applied to evaluate the environmental impacts of different industrial processes and identify ways to reduce waste.\\

The combination of these lectures offered high quality examples of novel research in systems engineering across many domains. A diverse range of topics was covered, including environmental engineering, electricity markets, agricultural technology, and smart buildings. Despite what might appear as a divergence in disciplines, a common theme emerged - the importance of interdisciplinary research in tackling global challenges such as climate change, sustainable agriculture, public health, urbanization, and energy. We were able to see that integrating community input and technical systems can ensure that solutions are tailored to meet the specific needs of local communities, closed-loop systems can minimize waste and maximize resource usage, and mathematical and computational techniques can be applied to optimize decision-making and manage risk.\\

Although there were many other specific highlights worth mentioning, the broader importance of interdisciplinary research was a consistent theme throughout the 10 presentations in Cornell Systems Seminar Series during the Spring 2023 semester. All of the lectures emphasized the need for collaboration between different fields of study to develop novel solutions to hard problems. Recall that in the opening presentation, Dr. Michael Charles emphasized the need for socio-technical systems that integrate community-based input with computational modeling. This was a good example of the broader emphasis placed on integrating diverse expertise, techniques, and perspectives, critical for proposing meaningful solutions that may likely not be possible with a classic disciplined approach. These lectures underscored the significance of interdisciplinary research and systems thinking to develop robust solutions to tackle the many global challenges that effect us all.\\
\end{document}