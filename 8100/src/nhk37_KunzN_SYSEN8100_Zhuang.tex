\documentclass[11pt]{article}
\pagenumbering{gobble} % suppress page number
\hyphenchar\font=-1 % suppress hyphenation
\setlength\parindent{0pt} % suppress indentation

\usepackage[margin=1truein]{geometry}
\usepackage{url}
\usepackage{xcolor}
\usepackage{hyperref}

\hypersetup{
    colorlinks = true,
    linkcolor = red,
    urlcolor = red
}

\begin{document}
\title{SYSEN 8100-1: Systems Seminar Series\\~\\
    \Large Houlong Zhuang: Materials Design in the Information Age and Beyond
}
\author{
    Nick Kunz [NetID: \url{nhk37}] \hyperlink{nhk37@cornell.edu}{nhk37@cornell.edu}
}
\date{September 30, 2022}
\maketitle

On September 30, 2022, Dr. Houlong Zhuang, Assistant Professor at the School for Engineering of Matter, Transport and Energy at Arizona State University, presented a case for applying state of the art technologies such as deep learning and quantum computing to improve the ways in which materials are designed.\\

Dr. Zhuang first highlighted the importance of High Entropy Materials (HEM) and the challenges associated with engineering them. He summarized the strength–ductility trade-off and described various ways that HEM's have historically been developed. This was important context, as it provided a basis for the novelty of his work regarding applied deep learning methods. One of the future directions that his work had pointed to was explainable AI, where he detailed the methods used to address the problem through feature removal and retesting. \\ 

Another important highlight that Dr. Zhuang emphasized was the role of quantum computing in the future development of HEM's. He broadly described the field as a context for the ways in which his lab has begun to apply it. The quantum encodings he detailed seem to contain the possibility of predicting atomic structures. Given that the limits of classical computing are likely to be confronted within the next decade, his lab's work could serve as important precedent as a way to address the limitations of classical computing for developing stronger materials.\\

It was interesting to learn the ways in which deep learning and quantum computing has been applied to material design. The interdisciplinary foundation of his research and the novel ways in which he was able to apply deep learning and quantum computing to a different vertical of engineering was a good example of the role that systems engineering could play in broader and integrated approaches to engineering at large. 

\end{document}