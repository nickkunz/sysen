\documentclass[11pt]{article}
\pagenumbering{gobble} % suppress page number
\hyphenchar\font=-1 % suppress hyphenation

\usepackage[margin=1truein]{geometry}
\usepackage{url}
\usepackage{xcolor}
\usepackage{hyperref}

\hypersetup{
    colorlinks = true,
    linkcolor = red,
    urlcolor = red
}

\begin{document}
\title{SYSEN 8100-1: Systems Seminar Series\\~\\
    \Large Barry Smith: The Ontology of Complex Systems and the Limits of Artificial Intelligence
}
\author{
    Nick Kunz [NetID: \url{nhk37}] \hyperlink{nhk37@cornell.edu}{nhk37@cornell.edu}
}
\date{\today}
\maketitle

\noindent On September 2, 2022, Dr. Barry Smith, Professor of Philosophy and Computer Science at the University of Buffalo, and Director of the National Center for Ontological Research, presented an argument on the limitations of artificial intelligence (AI).\\

\noindent In his presentation, Dr. Smith addressed many important topics related to AI and its relationship to systems. A foundational topic he introduced was basic systems theory. He outlined the differences between simple systems and complex systems, and how that related to the limitations of AI. He suggested that in its current state of development, AI can only effectively address specific and largely stationary problems - those within the known boundaries of a simple system. Here, Dr. Smith describes a simple system as those with predictable outcomes, such as solar systems, propulsion systems, networks, etc.\\

\noindent He goes on to describe complex systems and why it is likely the case that AI will never be able to fully address problems within these settings. He argued that because of the emergent behavior of complex systems, they are difficult, if not impossible to predict. Therefore, an AI within a complex system, cannot effectively operate. Dr. Smith further details complex systems as often having no theoretical bounds, largely non-stationary with arbitrary element combinations and interaction types, having force overlay, and a non-equilibrium state. \\

\noindent Although there were many important topics directly addressed in his presentation related to the limitations of AI, perhaps the most important one was an indirect suggestion. Dr. Smith highlighted the general idea that the real achievements of developing AI, were not the achievements of the systems, but rather the achievements of human capability. Perhaps we should be more focused on celebrating the human intellect, creativity, and collaboration that emerges from developing AI, rather than AI itself.

\end{document}