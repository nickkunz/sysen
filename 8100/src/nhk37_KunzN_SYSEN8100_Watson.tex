\documentclass[11pt]{article}
\pagenumbering{gobble} % suppress page number
\hyphenchar\font=-1 % suppress hyphenation
\setlength\parindent{0pt} % suppress indentation

\usepackage[margin=1truein]{geometry}
\usepackage{url}
\usepackage{xcolor}
\usepackage{hyperref}

\hypersetup{
    colorlinks = true,
    linkcolor = red,
    urlcolor = red
}

\begin{document}
\title{SYSEN 8100-1: Systems Seminar Series\\~\\
    \Large Michael Watson: Engineering Elegant Systems\\Engineering at the Systems Level
}
\author{
    Nick Kunz [NetID: \url{nhk37}] \hyperlink{nhk37@cornell.edu}{nhk37@cornell.edu}
}
\date{October 30, 2022}
\maketitle

On October 28, 2022, Dr. Michael Watson, Advanced Concepts Office Technical
Advisor for Systems Level Assessments at the Marshall Space Flight Center (MSFC) and retired Systems Engineer at NASA, presented his thoughts and analysis regarding systems engineering principles, concepts, and models, as well as broader implications related to policy and law. \\

Dr. Watson began by providing a 15 point description of various systems engineering principles, ranging from practical applications in law and policy, organizational behavior and leadership, as well as developments in systems theory and systems  thinking. He went on to describe his 3 systems engineering hypotheses regarding systems context, complexity, stakeholder preference, and category theory in mathematics. \\

He emphasized category theory as the formality in which higher level modeling can be conducted. The systems level modeling he described was in reference to a nested hierarchy of system model types. He detailed 6 system model types to include: 1) Relational (MBSE), 2) Physics-Based, 3) State Variable, 4) System Value, 5) Statistical, and 6) System Dynamics. He highlighted the importance that all of these are taken into consideration together, rather than as a single set. \\

The presentation also highlighted the importance of policy and legal considerations. The topics covered in this regard were broadly practical for use across system model types. They mainly addressed concerns related to specifying knowledge gaps, using consistent language, identifying opportunities for advancement, setting expected project duration, integrating engineering disciplines, and the impact of government oversight. \\

It was interesting to learn more about the different aspects of the 6 system model types. This becomes especially important in the context of currently learning MBSE. The details Dr. Watson provided regarding the mathematical formality of category theory were also interesting and provided a basis for future investigation on how these types of formulae might be applied to enrich future systems engineering research.

\end{document}