\documentclass[11pt]{article}
\pagenumbering{gobble} % suppress page number
\hyphenchar\font=-1 % suppress hyphenation
\setlength\parindent{0pt} % suppress indentation

\usepackage[margin=1truein]{geometry}
\usepackage{url}
\usepackage{xcolor}
\usepackage{hyperref}

\hypersetup{
    colorlinks = true,
    linkcolor = red,
    urlcolor = red
}

\begin{document}
\title{SYSEN 8100-1: Systems Seminar Series\\~\\
    \Large Leonardo Basso: Analytics Saved Lives \\During the COVID-19 Crisis in Chile
}
\author{
    Nick Kunz [NetID: \url{nhk37}] \hyperlink{nhk37@cornell.edu}{nhk37@cornell.edu}
}
\date{October 22, 2022}
\maketitle

On October 21, 2022, Dr. Leonardo Basso, Professor in the Department of Civil Engineering at Universidad de Chile and Director of the ANID Center of Excellence Instituto Sistemas Complejos de Ingeniería (ISCI), presented his supporting analysis for policy and decision making in Chile during the COVID-19 pandemic. He highlighted specific details regarding screening, vaccination, testing, and surveillance, as well as broader social impacts regarding healthcare, communications, and data privacy. \\

Dr. Basso described a number of important details of his analyses and their corresponding impacts. He emphasized how telecommunications systems and analytics were able to inform allocation decisions for critical healthcare services with limited capacity, such as intensive care, screening, and testing. Another important part of his analysis outlined the Chilean serology surveillance program, which addressed the need for useful information to create a national vaccination policy. \\

The presentation also highlighted the importance of private-public partnerships in addition to interdisciplinary teams coming together to focus on issues of national interest. It was clear that the pandemic response and overall public health in Chile benefited from the collaboration of many different disciplines and cooperation between institutions. Furthermore, that Chile continues to improve upon data-driven decision making and policy precedents established during the COVID-19 pandemic.\\

It was interesting to learn about how the strategic response to COVID-19 was implemented in Chile when compared to the response in the United States and the global community more broadly. Furthermore, how those strategies in Chile influenced others and what we might learn from them. Dr. Basso's presentation was a good example of the many useful applications of systems and analytics for public health and their importance in policy and decision making, especially during moments of uncertainty.

\end{document}