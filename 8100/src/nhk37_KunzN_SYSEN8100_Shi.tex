\documentclass[11pt]{article}
\pagenumbering{gobble} % suppress page number
\hyphenchar\font=-1 % suppress hyphenation
\setlength\parindent{0pt} % suppress indentation

\usepackage[margin=1truein]{geometry}
\usepackage{url}
\usepackage{xcolor}
\usepackage{hyperref}

\hypersetup{
    colorlinks = true,
    linkcolor = red,
    urlcolor = red
}

\begin{document}
\title{SYSEN 8100-1: Systems Seminar Series\\~\\
    \Large Rui Shi: Advancing Sustainability Through Systems Analysis
}
\author{
    Nick Kunz [NetID: \url{nhk37}] \hyperlink{nhk37@cornell.edu}{nhk37@cornell.edu}
}
\date{November 11, 2022}
\maketitle

On November 11, 2022, Dr. Rui Shi, Assistant Professor of Chemical Engineering at Pennsylvania State University and the Institutes of Energy and the Environment, presented her work related to Life Cycle Assessments (LCA) for sustainable energy systems. The focus of the presentation was on her recent work, which highlighted the trade-offs for a wide variety of possible scenarios when developing and integrating environmentally sustainable and economically feasible energy solutions.\\

Dr. Shi began with important context regarding LCA's for sustainable development and engineering. She outlined the 4 typical steps toward those ends, which were: 1) goal and scope, 2) inventory, 3) impacts, and 4) interpretation. Although these were explained mainly through chemical engineering examples, the broader and more important point was that there were many different pathways that these developments could take.\\

Given the many possible trajectories that each development could take, Dr. Shi highlighted the importance of LCA's that when applied, can often benefit from \textit{comparative} and \textit{systemic} approaches. One of the proposed solutions of her analysis introduced Agile LCA, a novel and important framework for characterizing the environmental impacts of developments with uncertain outcomes. \\

The presentation also highlighted how to measure and assess uncertainty in these scenarios. The focus in this regard was on sensitivity analysis and Monte Carlo simulation to determine which known variables have the largest impacts on green house gas emissions. Dr. Shi's work helped to illustrate how these methods can help shape useful solution spaces and aid in decision making for sustainable development by exhibiting critical trade-offs for given project goals.\\

It was interesting to learn the high variability and uncertainty in renewable energy systems. This was especially pronounced in the large variations in emissions output from bio-fuels. It was also interesting to learn about the broad flexibility that the Agile LCA framework affords through the examples given in the presentation. This begs the question how the Agile LCA framework might be applied in adjacent domains, such as urban planning and development.\\

\end{document}