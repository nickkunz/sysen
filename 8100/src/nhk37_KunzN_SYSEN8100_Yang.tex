\documentclass[11pt]{article}
\pagenumbering{gobble} % suppress page number
\hyphenchar\font=-1 % suppress hyphenation
\setlength\parindent{0pt} % suppress indentation

\usepackage[margin=1truein]{geometry}
\usepackage{url}
\usepackage{xcolor}
\usepackage{hyperref}

\hypersetup{
    colorlinks = true,
    linkcolor = red,
    urlcolor = red
}

\begin{document}
\title{SYSEN 8100-1: Systems Seminar Series\\~\\
    \Large Longqi Yang: Powering a Step-Function Change in Information Work
}
\author{
    Nick Kunz [NetID: \url{nhk37}] \hyperlink{nhk37@cornell.edu}{nhk37@cornell.edu}
}
\date{November 4, 2022}
\maketitle

On November 4, 2022, Dr. Longqi Yang, Strategic Applied Research Manager at Microsoft's Office of Applied Research, presented an overview of technological trends currently shaping today's workplace and organizations around the world. Dr. Yang argues that the rapid transformation we're experiencing with regard to remote and hybrid work warrant further investigation, as the data amassed from digital productivity platforms, such as Microsoft Teams, provide a rich corpus of information that can drive actionable insights to boost productivity.\\

Dr. Yang began with important context of the recent and growing phenomena of remote and hybrid work accelerated by the COVID-19 global pandemic. He mentioned that users spend roughly 2.8 billion minutes on the Microsoft Teams platform every day. Of those users, roughly 30\% are hybrid workers, 15\% are fully remote, and 65\% are on-site. What this meant was that there is a large amount of telemetry data, which can be utilized to infer greater meaning and drive decisions to better serve these combined modalities of work.\\

Another important topic of the presentation addressed an experience that many who conducted remote work during the COVID-19 pandemic could intimately relate to, which was multitasking. Dr. Yang had pointed to an important finding in the Microsoft Teams telemetry data, which suggested that multitasking had a greater likelihood of occurring during long meetings, recurring meetings, and early morning meetings. Furthermore, he detailed the specific types of multitasking, where roughly 30\% of multitasking was related to emails, and 25\% was related to file management.\\

The presentation also highlighted additional research conducted by his team in this regard. Data collected from approximately 62,000 Microsoft employees from December 2019 to June 2020, suggested that remote and hybrid work was driving more asynchronous communications, such as emails or messages, as well as longer hours worked per week by roughly 10\%. Dr. Yang had also emphasized the care taken to ensure that the data collected from employees was not misused to conduct workplace surveillance and to ensure anonymity. \\

It was interesting to learn the actionable insights that his team's research suggested for remote and hybrid work. The research suggested that in order to make synchronous meetings and communications more productive, that it was important to schedule shorter meetings, and to schedule breaks when longer meetings are required. It also suggested that early morning meetings for important discussion topics should be avoided. Remote and hybrid work appear to be here to stay, it will be interesting to see how productivity platforms continue to develop to serve this growing trend.\\

\end{document}