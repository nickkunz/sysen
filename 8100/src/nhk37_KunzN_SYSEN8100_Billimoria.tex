\documentclass[11pt]{article}
\pagenumbering{gobble} % suppress page number
\hyphenchar\font=-1 % suppress hyphenation
\setlength\parindent{0pt} % suppress indentation

\usepackage[margin=1truein]{geometry}
\usepackage{url}
\usepackage{xcolor}
\usepackage{hyperref}

\hypersetup{
    colorlinks = true,
    linkcolor = red,
    urlcolor = red
}

\begin{document}
\title{SYSEN 8100-1: Systems Seminar Series\\~\\
    \Large Farhad Billimoria: Fragility and Resilien in Low Carbon Power Systems, Physics-Informed Market Frameworks for System Security
}
\author{
    Nick Kunz [NetID: \url{nhk37}] \hyperlink{nhk37@cornell.edu}{nhk37@cornell.edu}
}
\date{October 18, 2022}
\maketitle

On October 14, 2022, Farhad Billimoria, PhD Candidate in the Department of Engineering Science at the University of Oxford, and Visiting Research Fellow with the Institute for Energy Studies, presented a case for energy market frameworks for energy system security and low-carbon energy sources. Broadly, the topic of his presentation was focused on the translation of engineering to economics and its policy implications in that regard. \\

Billimoria first described the likely transition to renewable energy sources given the policies that are currently being implemented around the world. He went on to address the engineering implications of those policies on energy system security, stability, and reliability. He also mentioned important real-world cases where these problems presented themselves. For example, the cascading and separation event in Australia in 2018, and the voltage dependency problem, more generally, \\

Billimoria highlighted the focus of his research based on market mechanisms for enabling energy system security. He introduced an important idea within his work that characterized energy markets and how they do not neatly categorize themselves between public and private market goods. Rather, it is more likely the case that they occupy a domain between the two, which he described as \textit{``club goods''}, and \textit{``common pool resources''}. \\

It was interesting to learn about the many new challenges that low-carbon and renewable energy sources introduce into energy system security, stability, and reliability. Furthermore, that there are adjacent market mechanisms for addressing engineering challenges. Billimoria's presentation was a sobering reminder that low-carbon and renewable energy sources are not a panacea for our sustainability goals and it will take careful planning, policy, engineering, and market participation to advance them in a way that maintains the security of our energy systems.

\end{document}