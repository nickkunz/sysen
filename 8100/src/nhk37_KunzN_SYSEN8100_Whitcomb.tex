\documentclass[11pt]{article}
\pagenumbering{gobble} % suppress page number
\hyphenchar\font=-1 % suppress hyphenation
\setlength\parindent{0pt} % suppress indentation

\usepackage[margin=1truein]{geometry}
\usepackage{url}
\usepackage{xcolor}
\usepackage{hyperref}

\hypersetup{
    colorlinks = true,
    linkcolor = red,
    urlcolor = red
}

\begin{document}
\title{SYSEN 8100-1: Systems Seminar Series\\~\\
    \Large Clifford Whitcomb: Competencies Uncovered – What Systems Engineers Need to Know
}
\author{
    Nick Kunz [NetID: \url{nhk37}] \hyperlink{nhk37@cornell.edu}{nhk37@cornell.edu}
}
\date{September 26, 2022}
\maketitle

On September 23, 2022, Dr. Clifford Whitcomb, Professor of Practice in Systems Engineering at Cornell University, presented relevant and important information regarding systems engineering competencies in the modern labor force. During his presentation, he surveyed many frameworks and models for evaluating the core competencies critical to the profession.\\

Dr. Whitcomb first described a unified definition of systems engineering competencies to include skills, knowledge, ability, behaviors, and other characteristics required for the job function. After which, he built upon that definition to survey several competency frameworks and models. Although there are potentially hundreds of frameworks and models, each with particular use cases in industry, the military, the academy, etc., he highlighted that there is no unified framework across a range of institutions. \\

He then went to revisit the fundamentals that were core to the occupation and summarized by the Engineering Competency Model (ECM) developed by the American Association of Engineering Societies (AAES) and the US Dept. of Labor (USDOL). Dr. Whitcomb summarized two important studies in that regard. The first was the US Navy's Systems Engineering Competency Model (SECCM), which identified 44 competencies and 179 nested tasks. The second was the INCOSE Systems Engineering Competency Framework and Competency Assessment Guide (SECF), which was different than the SECCM in that it emphasized a component of \textit{systems thinking}, and also attempted to unify the frameworks and models. \\

It was interesting to learn that the survey conducted as a part of the SECCM, found that the three most important skills that system engineers must have were: 1) communications, 2) problem solving, and 3) ethics. This goes against an intuitive sense that perhaps a technical skill would've been mentioned. However, the presentation exhibited a clear clustering of these skills with a bifurcation between them and all others. It begs the question how the value of these skills will change - if at all - in future studies and over a broader time-horizon.

\end{document}