\documentclass{article}

%% doc settings
\hyphenchar\font=-1 % suppress hyphenation
\setlength\parindent{0pt} % suppress indentation
\usepackage[margin=1.25truein]{geometry} % set page margins

%% libraries 
\usepackage{listings}
\usepackage{fancyhdr}
\usepackage{lastpage}
\usepackage{url}
\usepackage{xcolor}
\usepackage{hyperref}
\usepackage{amssymb}
\usepackage{amsthm}
\usepackage{amsmath}
\usepackage{algorithm}
\usepackage{algcompatible}
\usepackage{natbib}
\usepackage{tikz}
\usepackage{pgfplots}
\usepackage{textcomp}
\usepackage{subcaption}
\usetikzlibrary{shapes, arrows}

%% link viz
\hypersetup{
    colorlinks = true,
    linkcolor = red,
    urlcolor = red,
    citecolor = black
}

%% code colors 
\definecolor{codegreen}{rgb}{0,0.6,0}
\definecolor{codegray}{rgb}{0.5,0.5,0.5}
\definecolor{codepurple}{rgb}{0.58,0,0.82}
\definecolor{backcolour}{rgb}{0.95,0.95,0.92}

\lstdefinestyle{mystyle}{
    backgroundcolor=\color{backcolour},   
    commentstyle=\color{codegreen},
    keywordstyle=\color{magenta},
    numberstyle=\tiny\color{codegray},
    stringstyle=\color{codepurple},
    basicstyle=\ttfamily\footnotesize,
    breakatwhitespace=false,         
    breaklines=true,                 
    captionpos=b,                    
    keepspaces=true,                 
    numbers=left,                    
    numbersep=5pt,                  
    showspaces=false,                
    showstringspaces=false,
    showtabs=false,                  
    tabsize=2
}

\lstset{style=mystyle}

%% math ops
\DeclareMathOperator*{\argmax}{argmax} % thin space, limits underneath in displays

%% page nums
\pagestyle{fancy}
\fancyhf{}
\fancyfoot[C]{Pg. \thepage \space of \pageref*{LastPage}}
\renewcommand{\headrulewidth}{0pt}

%% begin doc
\begin{document}
\title{SYSEN 5200: Systems Analysis Behavior and Optimization\\~\\
    \Large Central Limit Theorem \& Reliability Analysis
}
\author{
    Nick Kunz [NetID: \url{nhk37}] \hyperlink{nhk37@cornell.edu}{nhk37@cornell.edu}}
\date{February 12, 2023}
\maketitle
\thispagestyle{fancy}

%% body doc
\begin{enumerate}

    \item Suppose that $X_1$ is a r.v. $N(3,15)$, and $X_2$ is a r.v. $N(10,6)$ independent from $X_1$. The probabilities of $P(X_1 + X_2 \geq z)$, as a standard normal can be expressed such that:\\

    $Z$ is a normal r.v. where:
    \begin{equation}
        \begin{split}
            \mu &= \mathbb{E}{[X_1 + X_2]}\\
            &= \mathbb{E}{[X_1]} + \mathbb{E}{[X_2]}\\
            &= 3+10\\
            &= 13
        \end{split}
    \end{equation}
    
    \begin{equation}
        \begin{split}
            \sigma^2 &= Var(X_1 + X_2)\\
            &= Var(X_1) + Var(X_2)\\
            &= 15+6\\
            &= 21
        \end{split}
    \end{equation}

    Therefore $Z \sim N(13, 21)$ such that:
    \begin{equation}
        \begin{split}
            P(X_1 + X_2 \geq z) &= P\Big(Z \geq \frac{z - \mu}{\sigma}\Big)\\
            &= P\Big(Z \geq \frac{z - 13}{\sqrt{21}}\Big)
        \end{split}
    \end{equation}\\

    \item Suppose that $X$ is a r.v. with mean 6, variance 10, and $X_1,\hdots,X_{100}$ are i.i.d. copies of $X$.

    \begin{enumerate}
        \item Applying the Central Limit Theorem to approximate $P\Big( \sum_{i=1}^{100}X_i \geq 1000\Big)$ we have:\\

        The same mean:
        \begin{equation}
            \begin{split}
                \Bar{X} &= \frac{1}{n} \sum_{i=1}^{n} X_i\\
                &= \frac{1}{100} \sum_{i=1}^{100} X_i\\
            \end{split}
        \end{equation}

        The probability:
        \begin{equation}
            \begin{split}
                P\Big( \sum_{i=1}^{100}X_i \geq 1000\Big) &= P\Big(100 \, \Bar{X} \geq 1000\Big)\\
                % &= P\Big(\Bar{X} \geq 10\Big)
            \end{split}
        \end{equation}

        The Central Limit Theorem:
        \begin{equation}
            \lim_{n \to \infty} \frac{\Bar{X}-\mu}{\sigma/\sqrt{n}} \xrightarrow{D} N(0, 1)
        \end{equation}

        when applied is:
        \begin{equation}
            \begin{split}
                P\Big( \sum_{i=1}^{100}X_i \geq 1000\Big) 
                % &= P\Big(\frac{\Bar{X}-\mu}{\sigma/\sqrt{100}} \geq \frac{10-\mu {\sigma/\sqrt{100}}\Big)\\
                &= P\Big(Z \geq \frac{1000 - 600}{\sqrt{1000}}\Big)\\
                &= P\Big(4\sqrt{10}\Big)
            \end{split}
        \end{equation}\\

        \item Suppose that $\sum_{i=1}^{n}X_i = 28$ and $n=100$. Applying the Central Limit Theorem to compute $(1 - \alpha)$ and approximate confidence intervals for $\alpha = 0.05$, we have:\\

        The same mean:
        \begin{equation}
            \begin{split}
                \Bar{X} &= \frac{1}{n} \sum_{i=1}^{n} X_i\\
                &= \frac{1}{100} \sum_{i=1}^{100} X_i\\
                &= \frac{28}{100} = 0.28\\
            \end{split}
        \end{equation}

        The approximate Confidence Interval:
        
        \begin{equation}
            \begin{split}
                P\Big(-Z\alpha/2 \leq \frac{\Bar{X}-\mu}{\sigma/\sqrt{n}} \leq Z\alpha/2\Big) &\approx 1-\alpha\\
                P\Big(\Bar{X}-Z\alpha/2 \cdot \sigma/\sqrt{n} \leq \mu \leq \Bar{X}+Z\alpha/2 \cdot \sigma/\sqrt{n}\Big) &\approx 1-\alpha\\ &\approx 0.95\\
            \end{split}
        \end{equation}
        
        and that:
        \begin{equation}
            Z\alpha/2 = 1.96
        \end{equation}\\
        when solved is:
        \begin{equation}
            \begin{split}
                & \Big(\Bar{X}\pm1.96 \cdot \sigma/\sqrt{n} \Big)\\
                & \Big(0.28 \pm1.96 \cdot \sigma/10 \Big)\\
                & \Big(0.28 \pm 0.14\Big)
            \end{split}
        \end{equation}
    \end{enumerate}

\newpage
    \item Considering a system where the lifetimes of the first and second components are geometrically distributed with mean 2 hours ($Geo(1/2)$), the lifetime of the third component is geometrically distributed with mean 4 hours ($Geo(1/4)$) and the lifetime of the fourth component is geometrically distributed with mean 3 hour ($Geo(1/3)$).

    \begin{enumerate}
        \item The structure function of this system is such that:
        \begin{equation}
            \begin{split}
                \phi(x) &= max(x_4, min(x_3, max(x_2, x_1))))\\
                &= 1 - (1-x_4) (1-x_3(1-(1-x_2)(1-x_1)))
            \end{split}
        \end{equation}\\
        \item The probability that the system survives for more than 2 hours is such that:
        \begin{equation}
            \begin{split}
                \mathbb{P}{(L>2)} &= (1-\lambda_1)^{t} (1-\lambda_3)^{t} + (1-\lambda_2)^{t} (1-\lambda_3)^{t} + (1-\lambda_4)^{t}\\
                & - (1-\lambda_1)^{t} (1-\lambda_2)^{t} (1-\lambda_3)^{t} (1-\lambda_4)^{t}\\
                &= L_1L_3 + L_2L_3 + L_4 - L_1L_2L_3L_4\\
                % &= 0.25 \cdot 0.56 + 0.25 \cdot 0.56 + 0.44 - 0.25 \cdot 0.25 \cdot 0.56 \cdot 0.44\\
                &= 0.71
            \end{split}
        \end{equation}
        % \begin{equation}
        %     \begin{split}
        %         \mathbb{P}{(L>2)} &= (L_4 L_3 L_1) + (L_4 L_3 L_2) - (L_4 L_3 L_2 L_1)\\
        %         &= \Big(\frac{4}{9} \cdot \frac{9}{16} \cdot \frac{1}{4}\Big) + \Big(\frac{4}{9} \cdot \frac{9}{16} \cdot \frac{1}{4}\Big) - \Big(\frac{4}{9} \cdot \frac{9}{16} \cdot \frac{1}{4} \cdot \frac{1}{4}\Big)\\
        %         &=\frac{7}{64}\\
        %         &\approx0.11
        %     \end{split}
        % \end{equation}
        where:
        \begin{equation}
            \begin{split}
                L_1 = L_2 &\sim \Big(1 - \frac{1}{2}\Big)^2\\
                &= \frac{1}{4}\\
                L_3 &\sim \Big(1 - \frac{1}{4}\Big)^2\\
                &= \frac{9}{16}\\
                L_4 &\sim \Big(1 - \frac{1}{3}\Big)^2\\
                &= \frac{4}{9}\\
            \end{split}
        \end{equation}\\
        
        \item The expected lifetime of the system in hours using the geometric series is such that:

        \begin{equation}
            \begin{split}
                \mathbb{E}{[L]} &= \sum_{t=0}^{\infty}\mathbb{P}{(L > t})\\
                &= \frac{1}{1-(1-\lambda_1) (1-\lambda_3)} + \frac{1}{1-(1-\lambda_2) (1-\lambda_3)} + \frac{1}{1-(1-\lambda_4)}\\
                &- \frac{1}{1-(1-\lambda_1) (1-\lambda_2) (1-\lambda_3) (1-\lambda_4)}\\
                % &=\frac{8}{5}+\frac{8}{5}+\frac{3}{1}-\frac{8}{7}\\
                &=3.45
            \end{split}
        \end{equation}
    
%% end
\end{enumerate}
\end{document}