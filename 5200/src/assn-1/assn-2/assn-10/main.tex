\documentclass{article}

%% doc settings
\hyphenchar\font=-1 % suppress hyphenation
\setlength\parindent{0pt} % suppress indentation
\usepackage[margin=1.25truein]{geometry} % set page margins

%% libraries 
\usepackage{listings}
\usepackage{fancyhdr}
\usepackage{lastpage}
\usepackage{url}
\usepackage{xcolor}
\usepackage{hyperref}
\usepackage{amssymb}
\usepackage{amsthm}
\usepackage{amsmath}
\usepackage{algorithm}
\usepackage{algcompatible}
\usepackage{natbib}
\usepackage{tikz}
\usepackage{pgfplots}
\usepackage{textcomp}
\usepackage{subcaption}
\usepackage{graphicx}
\usetikzlibrary{shapes, arrows.meta, positioning, calc}

%% link viz
\hypersetup{
    colorlinks = true,
    linkcolor = red,
    urlcolor = red,
    citecolor = black
}

%% code colors 
\definecolor{codegreen}{rgb}{0,0.6,0}
\definecolor{codegray}{rgb}{0.5,0.5,0.5}
\definecolor{codepurple}{rgb}{0.58,0,0.82}
\definecolor{backcolour}{rgb}{0.95,0.95,0.92}

\lstdefinestyle{mystyle}{
    backgroundcolor=\color{backcolour},   
    commentstyle=\color{codegreen},
    keywordstyle=\color{magenta},
    numberstyle=\tiny\color{codegray},
    stringstyle=\color{codepurple},
    basicstyle=\ttfamily\footnotesize,
    breakatwhitespace=false,         
    breaklines=true,                 
    captionpos=b,                    
    keepspaces=true,                 
    numbers=left,                    
    numbersep=5pt,                  
    showspaces=false,                
    showstringspaces=false,
    showtabs=false,                  
    tabsize=2
}

\lstset{style=mystyle}

%% math ops
\DeclareMathOperator*{\argmax}{argmax} % thin space, limits underneath in displays

%% page nums
\pagestyle{fancy}
\fancyhf{}
\fancyfoot[C]{Pg. \thepage \space of \pageref*{LastPage}}
\renewcommand{\headrulewidth}{0pt}

%% begin doc
\begin{document}
\title{SYSEN 5200: Systems Analysis Behavior and Optimization\\~\\
    \Large Dynamic Programming
}
\author{
    Nick Kunz [NetID: \url{nhk37}] \hyperlink{nhk37@cornell.edu}{nhk37@cornell.edu}}
\date{April 30, 2023}
\maketitle
\thispagestyle{fancy}

%% body doc
\begin{enumerate}

    \item Suppose that you have $N$ tickets available for a particular concert. On day $t$, a customer arrives and declares that she is willing to pay $P_t$ dollars for the ticket. You must decide to sell the ticket for $P_t$ or to decline the sale. Suppose that as the date of the concert nears, customers are willing to pay a higher cost for the ticket, so that the distribution of $P_t$ is sampled according to:
    \begin{equation}
        \begin{split}
            P(P_t = p) =
            \begin{cases}
                \frac{t}{2T} & \text{for } p = 20\\
                \frac{T - t}{2T} & \text{for } p = 10\\
                \frac{1}{2} & \text{for } p = 5
            \end{cases}
        \end{split}
    \end{equation}

    The concert is on the evening of day $T$, thus any tickets not sold by the end of day $T$ will go to waste. Let $V(t, n)$ denote the expected profit from an optimal ticket sales strategy after (and including) day $t$ if you have $n$ tickets left at the beginning of day $t$. Using dynamic programming to calculate $V(1,N)$, we compute the optimal strategy for managing ticket sales.\\

    \begin{enumerate}
        \item To compute the expected sales profit for the base cases $V(T+1,n)$, we have:
        \begin{equation}
            V(T+1, n) = 0 \, \forall \, n\in[0, N]
        \end{equation}
        
        \item The recursive formula for computing $V(t,n)$ is:
        \begin{equation}
            V(t, n) = \max\left(\sum_{p} P(P_t=p) \cdot \max(p + V(t+1, n-1), V(t+1, n)), V(t+1, n)\right)
        \end{equation}

        \item The pseudo-code for the dynamic program can be written as:
        \begin{algorithm}[H]
        \caption{Optimal Ticket Sales}
        \begin{algorithmic}[1]
        \STATE \textbf{Init:} $V = (T+1) \times (N+1)$
        \FOR{$t = T$ \textbf{down to} $1$}
        \FOR{$n = 0$ \textbf{to} $N$}
        \STATE \text{not sell }$\gets V[t+1][n]$
        \STATE $\text{sell } \gets 0$
        \FOR{$p \in {5, 10, 20}$}
        \IF{$n > 0$}
        \STATE $P_{t} \gets 0$
        \IF{$p = 5$}
        \STATE $P_{t} \gets \frac{1}{2}$
        \ELSIF{$p = 10$}
        \STATE $P_{t} \gets \frac{T - t}{2T}$
        \ELSIF{$p = 20$}
        \STATE $P_{t} \gets \frac{t}{2T}$
        \ENDIF
        \STATE $\text{sell } \gets \text{sell } + P_{t} \cdot \max(p + V[t+1][n-1], V[t+1][n])$
        \ENDIF
        \ENDFOR
        \STATE $V[t][n] \gets \max(\text{sell}, \text{not sell})$
        \ENDFOR
        \ENDFOR
        \STATE \textbf{Output:} $V[1][N] = V(1, N)$
        \end{algorithmic}
        \end{algorithm}
        
        \item Solving for the setting $N = 3$ and $T = 10$, we have:
        \begin{lstlisting}[language=Python, title=Fig. Python 1(d)]
## optimal ticket sales
def optim_ticket_sales(N, T):

    ## init
    V = [[0 for i in range(N + 1)] for i in range(T + 2)]

    ## recursion
    for t in range(T, 0, -1):
        for n in range(N + 1):
            not_sell = V[t + 1][n]
            sell = 0
            for p in [5, 10, 20]:
                if n > 0:
                    if p == 5:
                        P_t = 1 / 2
                    elif p == 10:
                        P_t = (T - t) / (2 * T)
                    elif p == 20:
                        P_t = t / (2 * T)
                    sell += (
                        P_t * max(p + V[t + 1][n - 1], V[t + 1][n])
                    )
            V[t][n] = max(sell, not_sell)
    return V, V[1][N]

## results
N = 3
T = 10
V, profit = optim_ticket_sales(N, T)
print("Expected Profit: $", round(profit, 2))
\end{lstlisting}

        Expected Profit: \textbf{\$50.71}\\

        \item A table with the values computed for $V(t, n)$ of all values $t$ from 1 to 10, as well as $n$ from 0 to 3, is such that:
        \begin{lstlisting}[language=Python, title=Fig. Python 1(e)]
## table of values
def table_values(V, N, T):
    column = "t\\n | "
    for n in range(N + 1):
        column += f"{n} | "
    print(column)
    print("-" * len(column))
    for t in range(1, T + 1):
        row = f"{t} | "
        for n in range(N + 1):
            row += f"{V[t][n]:.2f} | "
        print(row)

## results
N = 3
T = 10
V, profit = optim_ticket_sales(N, T)
table_values(V, N, T)
\end{lstlisting}

        \begin{center}
            \begin{tabular}{c|c|c|c|c}
            $V(t,n)$ & 0 & 1 & 2 & 3 \\ \hline
            1 & 0.00 & 19.51 & 36.93 & 50.71 \\
            2 & 0.00 & 19.48 & 36.79 & 50.39 \\
            3 & 0.00 & 19.43 & 36.50 & 49.71 \\
            4 & 0.00 & 19.32 & 36.00 & 48.60 \\
            5 & 0.00 & 19.16 & 35.21 & 46.95 \\
            6 & 0.00 & 18.87 & 33.99 & 44.60 \\
            7 & 0.00 & 18.39 & 32.11 & 41.09 \\
            8 & 0.00 & 17.52 & 29.19 & 36.00 \\
            9 & 0.00 & 15.88 & 24.50 & 24.50 \\
            10 & 0.00 & 12.50 & 12.50 & 12.50 \\
            \end{tabular}
        \end{center}
        \,\\
        
        \item The optimal strategy can be derived from the table of $V(t, n)$ values in part 1(e). Notice that when $n = 1$, the expected profit increases as the concert date approaches, reaching its peak at \$19.51 on day 1. For $n = 2$ or $n = 3$, the expected profit also increases as the concert date approaches, with the highest expected profit of \$36.93 for $n = 2$ and \$50.71 for $n = 3$ on day 1.\\

        Therefore, if you have 1 ticket left, it is generally better to wait and sell the ticket closer to the concert date as the expected profit increases. If you have 2 or 3 tickets left, a similar strategy applies, with a preference for selling the tickets at a higher price as the concert date nears. This strategy takes advantage of the increasing probability of receiving a higher price as the concert date approaches while considering the available tickets to sell.
    \end{enumerate}

%% end
\end{enumerate}
\end{document}