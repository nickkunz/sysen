\documentclass{article}

%% doc settings
\hyphenchar\font=-1 % suppress hyphenation
\setlength\parindent{0pt} % suppress indentation
\usepackage[margin=1.25truein]{geometry} % set page margins

%% libraries 
\usepackage{listings}
\usepackage{fancyhdr}
\usepackage{lastpage}
\usepackage{url}
\usepackage{xcolor}
\usepackage{hyperref}
\usepackage{amssymb}
\usepackage{amsthm}
\usepackage{amsmath}
\usepackage{algorithm}
\usepackage{algcompatible}
\usepackage{natbib}
\usepackage{tikz}
\usepackage{pgfplots}
\usepackage{textcomp}
\usepackage{subcaption}
\usetikzlibrary{shapes, arrows}

%% link viz
\hypersetup{
    colorlinks = true,
    linkcolor = red,
    urlcolor = red,
    citecolor = black
}

%% code colors 
\definecolor{codegreen}{rgb}{0,0.6,0}
\definecolor{codegray}{rgb}{0.5,0.5,0.5}
\definecolor{codepurple}{rgb}{0.58,0,0.82}
\definecolor{backcolour}{rgb}{0.95,0.95,0.92}

\lstdefinestyle{mystyle}{
    backgroundcolor=\color{backcolour},   
    commentstyle=\color{codegreen},
    keywordstyle=\color{magenta},
    numberstyle=\tiny\color{codegray},
    stringstyle=\color{codepurple},
    basicstyle=\ttfamily\footnotesize,
    breakatwhitespace=false,         
    breaklines=true,                 
    captionpos=b,                    
    keepspaces=true,                 
    numbers=left,                    
    numbersep=5pt,                  
    showspaces=false,                
    showstringspaces=false,
    showtabs=false,                  
    tabsize=2
}

\lstset{style=mystyle}

%% math ops
\DeclareMathOperator*{\argmax}{argmax} % thin space, limits underneath in displays

%% page nums
\pagestyle{fancy}
\fancyhf{}
\fancyfoot[C]{Pg. \thepage \space of \pageref*{LastPage}}
\renewcommand{\headrulewidth}{0pt}

%% begin doc
\begin{document}
\title{SYSEN 5200: Systems Analysis Behavior and Optimization\\~\\
    \Large Probability \& Statistics: Random Variables
}
\author{
    Nick Kunz [NetID: \url{nhk37}] \hyperlink{nhk37@cornell.edu}{nhk37@cornell.edu}}
\date{Feburary 5, 2023}
\maketitle
\thispagestyle{fancy}

%% body doc
\begin{enumerate}

    \item Suppose $X$ is a continuous r.v. with the following p.d.f.
    
    \begin{equation}\nonumber
    f(x) = \begin{cases}
        cx^{-5} & \text{for } 2 \leq x < \infty\\
        0 & \text{for } x < 2
        \end{cases}
    \end{equation}\\

    \begin{enumerate}
        \item The constant $c$ can be computed as:
        
        \begin{equation}
            \begin{split}
                \int_{-\infty}^{\infty} f(x) \,dx &= \int_{2}^{\infty} cx^{-5} \,dx = 1\\
                &= cx^{-5} \bigg|_{2}^{\infty}\\
                &= \frac{c}{64} \Rightarrow c = 64
            \end{split}
        \end{equation}
        
        \item The c.d.f. can be computed as:
        
        \begin{equation}
            \begin{split}
                F(x) &= \int_{-\infty}^{x} f(y) \,dy\\
                &= \int_{2}^{x} 64y^{-5} \,dy\\
                &= 64y^{-5} \bigg|_{2}^{x}\\
                &= 64[-1/4y^{4}] \bigg|_{2}^{x}\\
                &= \begin{cases}
                    1 - (16/x^4) & \text{ for } x \geq 2\\
                    0 & \text{ for } x < 2
                \end{cases}
            \end{split}
        \end{equation}

\newpage
        \item The $\mathbb{E}{[X]}$ and $Var(X)$ can be computed as:
        
        \begin{equation}
            \begin{split}
            \mathbb{E}{[X]} & = \int_{-\infty}^{\infty}x \,f(x) \, dx \\
            &= \int_{2}^{\infty} x \, (64x^{-5}) \, dx \\
            &= \frac{8}{3}\\~\\
            \mathbb{E}{[X^{2}]} & = \int_{-\infty}^{\infty}x^{2} \,f(x) \, dx \\
            &= \int_{2}^{\infty} x^{2} \, (64x^{-5}) \, dx \\
            &= 8\\~\\
            Var(X) &= \mathbb{E}{[X^2]} - (\mathbb{E}{[X])^2}\\
            &= 8 - (8/3)^2\\
            &= \frac{8}{9}
            \end{split}
        \end{equation}
        
        \item The $\mathbb{E}{[X^4]}$ can be computed as:
        \begin{equation}
            \begin{split}
            \mathbb{E}{[X^{4}]} & = \int_{-\infty}^{\infty}x^{4} \,f(x) \, dx \\
            &= \int_{2}^{\infty} x^{4} \, (64x^{-5}) \, dx \\
            &= \infty
            \end{split}
        \end{equation}
    \end{enumerate}

    \item Suppose that $X$ is a continuous uniform r.v. with p.d.f. $U[10, 17]$. $P(X \geq 15)$ can be computed by evaluating all integrals with the c.d.f.:

    \begin{equation}
        F(x) = \begin{cases}
        0 & \text{for } x < a\\
        \frac{x -a }{b - a} & \text{for } a \leq x \leq b\\
        1 & \text{for } x > b
        \end{cases}
    \end{equation}

    where:
    \begin{equation}
        U[a,b]
    \end{equation}\\

    Hence:
    \begin{equation}
        \begin{split}
            P(X \geq 15) &= 1 - F(x)\\
            &= 1 - (5/7)\\
            &= \frac{2}{7}
        \end{split}
    \end{equation}
    
    where:
    \begin{equation}
        F(15) = \frac{15-10}{17-10} = \frac{5}{7}
    \end{equation}

    \item Suppose that $X$ is a continuous r.v. with p.d.f. $Expo(\lambda)$, and $a$ is some positive number such that $a < \frac{\lambda}{2}$. $Var(e^{aX})$ can be computed by evaluating all integrals.\\

    Given the p.d.f. of the exponential r.v. $X$:
    
    \begin{equation}
        \begin{split}
            f(x) = \begin{cases}
            \lambda e^{-\lambda x} & x > 0\\
            0 & \text{otherwise}
            \end{cases}
        \end{split}
    \end{equation}\\

    The $Var(e^{aX})$ can be computed as:
    
    \begin{equation}
        \begin{split}
        \mathbb{E}{[e^{aX}]} & = \int_{-\infty}^{\infty}x \,f(x) \, dx \\
        &= \int_{0}^{\infty} e^{aX} \, \lambda e^{-\lambda x} \, dx \\
        &= \lambda \bigg[\frac{e^{-(\lambda-a)x}}{-(\lambda-a)} \bigg]_{0}^{\infty}\\
        &= \frac{\lambda}{(\lambda-a)}\\~\\
        \mathbb{E}{[e^{2aX}]} & = \int_{-\infty}^{\infty}x^{2} \,f(x) \, dx \\
        &= \int_{0}^{\infty} e^{2aX} \, \lambda e^{-\lambda x} \, dx \\
        &= \lambda \bigg[\frac{e^{-(\lambda-2a)x}}{-(\lambda-2a)} \bigg]_{0}^{\infty}\\
        &= \frac{\lambda}{(\lambda-2a)}\\~\\
        Var(e^{aX}) &= \mathbb{E}{[e^{2aX}]} - (\mathbb{E}{[e^{aX}])^2}\\
        &= (\lambda / \lambda - 2a) - (\lambda/(\lambda-a))^{2}\\
        &= \frac{\lambda}{(\lambda-2a)} - {\frac{\lambda^{2}}{(\lambda-a)^{2}}}
        \end{split}
    \end{equation}\\~\\

    \item Let $X$ and $Y$ be Bernoulli r.v.'s with $P(X = 1 \text{ and } Y = 1) = 1/3$ with the probability of success $p$ and $q$ respectively, such that:\\

    \begin{equation}\nonumber
    X = \begin{cases}
        1 & \text{with probability } p\\
        0 & \text{with probability } 1-p
        \end{cases},\\~\\
    Y = \begin{cases}
        1 & \text{with probability } q\\
        0 & \text{with probability } 1-q
        \end{cases}
    \end{equation}\\

    \begin{enumerate}
        \item The $Cov(X, Y)$ can be computed as:

        \begin{equation}
            \begin{split}
                Cov(X,Y) &= \mathbb{E}{\big[(X-\mathbb{E}{[X]}) - (Y-\mathbb{E}{[Y]})\big]}\\
                &= \mathbb{E}{[XY]} - \mathbb{E}{[X]}\mathbb{E}{[Y]}\\
                &= \frac{1}{3} - p \cdot q
            \end{split}
        \end{equation}

        where:
        \begin{equation}
            \begin{split}
                \mathbb{E}{[X]} &= p\\
                \mathbb{E}{[Y]} &= q\\~\\
                \mathbb{E}{[XY]} &= P(X = 1 \text{ and } Y = 1)\\
                &= P(X=1) P(Y = 1|X = 1)\\
                &= \frac{1}{3}
            \end{split}
        \end{equation}
        
        \item The values of $X$ and $Y$ uncorrelated when $p$ and $q$ are:

        \begin{equation}
            \begin{split}
                Cov(X,Y) &= \frac{1}{3} - p \cdot q\\
                &= \frac{1}{3} - \frac{1}{3}\\
                &= 0
            \end{split}
        \end{equation}

        where:
        \begin{equation}
            p \cdot q = \frac{1}{3}
        \end{equation}\\

        \item $X$ and $Y$ \textit{are} independent when $p \cdot q = 1/3$, as we have $P(X=1 \text{ and } Y=1) = P(X=1) P(Y=1)$, such that:
        \begin{equation}
        \begin{split}
            P(X=1 \text{ and } Y=1) &= \frac{1}{3}\\
            &= p \cdot q
        \end{split}
        \end{equation}
    \end{enumerate}
    
%% end
\end{enumerate}
\end{document}