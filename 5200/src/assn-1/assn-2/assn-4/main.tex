\documentclass{article}

%% doc settings
\hyphenchar\font=-1 % suppress hyphenation
\setlength\parindent{0pt} % suppress indentation
\usepackage[margin=1.25truein]{geometry} % set page margins

%% libraries 
\usepackage{listings}
\usepackage{fancyhdr}
\usepackage{lastpage}
\usepackage{url}
\usepackage{xcolor}
\usepackage{hyperref}
\usepackage{amssymb}
\usepackage{amsthm}
\usepackage{amsmath}
\usepackage{algorithm}
\usepackage{algcompatible}
\usepackage{natbib}
\usepackage{tikz}
\usepackage{pgfplots}
\usepackage{textcomp}
\usepackage{subcaption}
\usetikzlibrary{shapes, arrows}

%% link viz
\hypersetup{
    colorlinks = true,
    linkcolor = red,
    urlcolor = red,
    citecolor = black
}

%% code colors 
\definecolor{codegreen}{rgb}{0,0.6,0}
\definecolor{codegray}{rgb}{0.5,0.5,0.5}
\definecolor{codepurple}{rgb}{0.58,0,0.82}
\definecolor{backcolour}{rgb}{0.95,0.95,0.92}

\lstdefinestyle{mystyle}{
    backgroundcolor=\color{backcolour},   
    commentstyle=\color{codegreen},
    keywordstyle=\color{magenta},
    numberstyle=\tiny\color{codegray},
    stringstyle=\color{codepurple},
    basicstyle=\ttfamily\footnotesize,
    breakatwhitespace=false,         
    breaklines=true,                 
    captionpos=b,                    
    keepspaces=true,                 
    numbers=left,                    
    numbersep=5pt,                  
    showspaces=false,                
    showstringspaces=false,
    showtabs=false,                  
    tabsize=2
}

\lstset{style=mystyle}

%% math ops
\DeclareMathOperator*{\argmax}{argmax} % thin space, limits underneath in displays

%% page nums
\pagestyle{fancy}
\fancyhf{}
\fancyfoot[C]{Pg. \thepage \space of \pageref*{LastPage}}
\renewcommand{\headrulewidth}{0pt}

%% begin doc
\begin{document}
\title{SYSEN 5200: Systems Analysis Behavior and Optimization\\~\\
    \Large Reliability Analysis
}
\author{
    Nick Kunz [NetID: \url{nhk37}] \hyperlink{nhk37@cornell.edu}{nhk37@cornell.edu}}
\date{February 21, 2023}
\maketitle
\thispagestyle{fancy}

%% body doc
\begin{enumerate}

    \item Consider the following system, where components one and two are parallel, which are both connected to three. Assume that the lifetime of the first component is uniformly distributed over $[0,1]$, the lifetime of the second component is uniformly distributed over $[0,2]$, and the lifetime of the third component is uniformly distributed over $[0,3]$ and is given by:
    
    \begin{equation}
        \phi(x)=1-x_3(1-(1-x_1)(1-x_2))
    \end{equation}\\
    \begin{enumerate}
        \item The probability that the lifetime of the system exceeds 0.25 time units is such that:

        \begin{equation}
            \begin{split}
                 \mathbb{P}{(L > t)} &= p_1(t) \cdot p_3(t) + p_2(t) \cdot p_3(t) - p_1(t) \cdot p_2(t) \cdot p_3(t)\\
                 &= 0.75 \cdot 0.917 + 0.875 \cdot 0.917 - 0.75 \cdot 0.875 \cdot 0.917\\
                 &\approx 0.88
            \end{split}
        \end{equation}

        where in the general case:
        \begin{equation}
            \begin{split}
                \mathbb{P}{(L_i > t)} &= \mathbb{I}{(t \leq b) \min{\Big(1, \frac{b-t}{b-a}}\Big)}\\~\\
                \mathbb{P}{(L_1 > 0.25)} &= \mathbb{I}{(0.25 \leq 1) \min{\Big(1, \frac{1-0.25}{1-0}}\Big)}\\
                & = \frac{0.75}{1} = 0.75\\~\\
                \mathbb{P}{(L_2 > 0.25)} &= \mathbb{I}{(0.25 \leq 2) \min{\Big(1, \frac{2-0.25}{2-0}}\Big)}\\
                & = \frac{1.75}{2} = 0.875\\~\\
                \mathbb{P}{(L_3 > 0.25)} &= \mathbb{I}{(0.25 \leq 3) \min{\Big(1, \frac{3-0.25}{3-0}}\Big)}\\
                & = \frac{2.75}{3} = 0.917
            \end{split}
        \end{equation}

        or more simply in this case:
        \begin{equation}
            \begin{split}
                \mathbb{P}{(L_i > t)} &= \Big(\frac{b-t}{b-a}\Big)\\
                \mathbb{P}{(L_1 > t)} &= \Big(\frac{1-0.25}{1}\Big)\\
                \mathbb{P}{(L_2 > t)} &= \Big(\frac{2-0.25}{2}\Big)\\
                \mathbb{P}{(L_3 > t)} &= \Big(\frac{3-0.25}{3}\Big)
            \end{split}
        \end{equation}\\

        \item The expected lifetime of the system is such that:

        \begin{equation}
            \begin{split}
                \mathbb{P}{(L \geq t)} = \begin{cases}
                    1-(1-t/3)(1-(1-(1-t/1))(1-(1-t/2))) & \text{ if } 0 \leq t \leq 1\\
                    1-(1-t/3)(1-(1-0)(1-(1-t/2))) & \text{ if } 1 \leq t \leq 2\\
                    1-(1-t/3)(1-(1-0)(1-0)) & \text{ if } 2 \leq t \leq 3\\
                    0 & \text{ if } t \geq 3\\
                \end{cases}
            \end{split}
        \end{equation}\\

        that when solved is:
        \begin{equation}
            \begin{split}
                \mathbb{E}{[L]} &= \int_{0}^{\infty}\mathbb{P}{(L \geq t)}\, dt\\
                &= \int_{0}^{1} 1-(1-t/3)(1-(1-(1-t/1))(1-(1-t/2))) \, dt \\
                &+ \int_{1}^{2} 1-(1-t/3)(1-(1-0)(1-(1-t/2))) \, dt \\
                &+ \int_{2}^{3} 1-(1-t/3)(1-(1-0)(1-0)) \, dt \\
                &\approx{0.84}
            \end{split}
        \end{equation}

        where:
        \begin{equation}
            \begin{split}
                \mathbb{P}{(L_i > t)} &= \begin{cases}
                    1-t/b & \text{ if } a \leq t \leq b\\
                    0 & \text{ if } t \geq b
                \end{cases}\\~\\
                \mathbb{P}{(L_1 > t)} &= \begin{cases}
                    1-t/1 & \text{ if } 0 \leq t \leq 1\\
                    0 & \text{ if } t \geq 1
                \end{cases}\\
                \mathbb{P}{(L_2 > t)} &= \begin{cases}
                    1-t/2 & \text{ if } 0 \leq t \leq 2\\
                    0 & \text{ if } t \geq 2
                \end{cases}\\
                \mathbb{P}{(L_3 > t)} &= \begin{cases}
                    1-t/3 & \text{ if } 0 \leq t \leq 3\\
                    0 & \text{ if } t \geq 3
                \end{cases}\\
            \end{split}
        \end{equation}
    \end{enumerate}

\newpage
    \item Suppose you want to learn how to bake during this quarantine, and the three important factors of a successful baking are ingredients, temperature and time. The ideal ingredients for the flour is cake flour, and you can also substitute it with a mixture of all-purpose flour and corn starch with certain ratio. Suppose you have sufficient cake flour w.p. $p_1$, and sufficient all purpose flour and sufficient corn starch for the substitute mixture w.p. $p_2$ and $p_3$. After successfully preparing the ingredients, you hope to set up the right temperate and cook time for the oven. However, unfortunately there is something wrong with the oven’s electronic display. Therefore w.p. $p_4$ you set the temperature correctly and w.p. $p_5$ you set the time appropriately.\\

    \begin{enumerate}
        \item Let $A$ denote the event that a cake was baked successfully. In other words, $A$ is a function of sub-events described above by composing nested intersections of five events $A_1, A_2, \hdots, A_5$, such that:

        \begin{equation}
            \begin{split}
                A &= (A_1 \cup (A_2 \cap A_3) \cap A_4 \cap A_5)
            \end{split}
        \end{equation}

        where:\\
        \begin{equation}
            \begin{split}
                A_1: & \text{Sufficient cake flour } p_1.\\
                A_2: & \text{Sufficient all-purpose flour } p_2.\\
                A_3: & \text{Sufficient corn starch } p_3.\\
                A_4: & \text{Correct temperature } p_4.\\
                A_5: & \text{Appropriate time } p_5.
            \end{split}
        \end{equation}\\

        \item $x_1, x_2, \hdots, x_5$ are indicator functions whether the events $A_1, A_2, \hdots, A_5$ are true, respectively. The structure function $\phi(x_1, \hdots, x_5)$ in terms of $x_i$'s is defined as follows (1 if event $A$ is true, i.e. successfully baked the cake, and 0 otherwise). 

        \begin{equation}
            \begin{split}
                \phi(x) &= \min(\max(x_1, \min(x_2, x_3)), x_4, x_5)\\
                &=  1-(1-x_1)(1-x_2 \, x_3) \, x_4 \, x_5
            \end{split}
        \end{equation}\\

        \item Assuming all the sub-events are independent, the probability of event $A$ is a function of $p_1, p_2,\hdots, p_5$ such that:

        \begin{equation}
            \begin{split}
                r(p) &= \mathbb{P}{[\phi(x)=1]}\\
                &= 1-(1-p_1)(1-p_2 \, p_3) \, p_4 \, p_5
            \end{split}
        \end{equation}\\

        \item Note that the independence assumption above may not be true. For example, it may be that many recipes use combinations of cake flour, all purpose flour, and corn starch such that the amount left in your pantry of these items tend to be dependent. The part of the computed probability derivation above that fails is that the availability of the ingredients which may be related such that:
        
        \begin{equation}
            \mathbb{P}{(x_2 \, x_3 = 1)} \neq \mathbb{P}{(x_2=1)} \mathbb{P}{(x_3=1)} = p_2 \, p_3 
        \end{equation}
        
    \end{enumerate}
%% end
\end{enumerate}
\end{document}