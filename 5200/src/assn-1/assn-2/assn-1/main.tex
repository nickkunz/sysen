\documentclass{article}

%% doc settings
\hyphenchar\font=-1 % suppress hyphenation
\setlength\parindent{0pt} % suppress indentation
\usepackage[margin=1.25truein]{geometry} % set page margins

%% libraries 
\usepackage{listings}
\usepackage{fancyhdr}
\usepackage{lastpage}
\usepackage{url}
\usepackage{xcolor}
\usepackage{hyperref}
\usepackage{amssymb}
\usepackage{amsthm}
\usepackage{amsmath}
\usepackage{algorithm}
\usepackage{algcompatible}
\usepackage{natbib}
\usepackage{tikz}
\usepackage{pgfplots}
\usepackage{textcomp}
\usepackage{subcaption}
\usetikzlibrary{shapes, arrows}

%% link viz
\hypersetup{
    colorlinks = true,
    linkcolor = red,
    urlcolor = red,
    citecolor = black
}

%% code colors 
\definecolor{codegreen}{rgb}{0,0.6,0}
\definecolor{codegray}{rgb}{0.5,0.5,0.5}
\definecolor{codepurple}{rgb}{0.58,0,0.82}
\definecolor{backcolour}{rgb}{0.95,0.95,0.92}

\lstdefinestyle{mystyle}{
    backgroundcolor=\color{backcolour},   
    commentstyle=\color{codegreen},
    keywordstyle=\color{magenta},
    numberstyle=\tiny\color{codegray},
    stringstyle=\color{codepurple},
    basicstyle=\ttfamily\footnotesize,
    breakatwhitespace=false,         
    breaklines=true,                 
    captionpos=b,                    
    keepspaces=true,                 
    numbers=left,                    
    numbersep=5pt,                  
    showspaces=false,                
    showstringspaces=false,
    showtabs=false,                  
    tabsize=2
}

\lstset{style=mystyle}

%% math ops
\DeclareMathOperator*{\argmax}{argmax} % thin space, limits underneath in displays

%% page nums
\pagestyle{fancy}
\fancyhf{}
\fancyfoot[C]{Pg. \thepage \space of \pageref*{LastPage}}
\renewcommand{\headrulewidth}{0pt}

%% begin doc
\begin{document}
\title{SYSEN 5200: Systems Analysis Behavior and Optimization\\~\\
    \Large Probability \& Statistics: Conditional Probability
}
\author{
    Nick Kunz [NetID: \url{nhk37}] \hyperlink{nhk37@cornell.edu}{nhk37@cornell.edu}}
\date{January 31, 2023}
\maketitle
\thispagestyle{fancy}

%% body doc
\begin{enumerate}
    \item In an experiment where we flip a fair coin 3 times and record the heads/tails sequence, $A$ is when we get an even number of heads (zero is not counted as even), $B$ is when we get at least 1 head. Therefore, the following can be computed as such:
    \begin{equation}\nonumber
    \begin{split}
        P(A) &= \frac{3}{8}\\
        P(B) &= \frac{7}{8}\\
        A^{c} &= \{\text{HHH, THT}\} = 2\\
        P(A^{c}) &= \frac{1}{4}\\
        B^{c} &= \{\text{TTT}\} = 1\\
        P(B^{c}) &= \frac{1}{8}\\
        A \cup B &= \{\text{HHH, HHT, HTH, THH, HTT, THT, TTH, HHT, HTH, THH}\} = 7\\
        A \cap B &= \{\text{HHT, HTH, THH}\} = 3\\
        P(A \cap B) &= \frac{1}{4}\\
        P(A \cup B) &= \frac{7}{8}\\
        (A^{c})^{c} &= \{\text{HHT, HTH, THH}\} = 3\\
        P(A^{c})^{c} &= \frac{3}{8}\\
        (A \cup B)^{c} &= \{\text{TTT}\} = 1\\
        A^{c} \cap B^{c} &= \{\text{}\} = 0\\
        P((A \cup B)^{c}) &= \frac{1}{8}\\
        A^{c} \cup B^{c} &= \{\text{TTH, THT, HTT, TTT}\} = 4\\
        P((A \cap B)^{c}) &= 0\\
    \end{split}
    \end{equation}

\newpage
    \item Suppose there has been an outbreak of a new flu virus. It is known that 8\% of the population has this flu. Suppose that there is a certain test for this flu. Conditional on a person having the flu, there is a 98\% probability that the test returns positive. Conditional on a person not having the flu, there is a 3\% probability that the test returns positive.

    \begin{enumerate}
        \item The conditional probability that a patient is actually infected given they test positive:\\

        Given Bayes Theorem:

        \begin{equation}
            P(A|B) = \frac{P(B|A) P(A)}{P(B)}
        \end{equation}\\

        Let the following be true:

        \begin{equation}
        \begin{split}
            \text{Event } A &= \text{Infected}\\
            \text{Event } A' &= \text{Not Infected}\\
            \text{Event } B &= \text{Positive Test}\\
            \text{Event } B' &= \text{Negative Test}\\
        \end{split}
        \end{equation}

        so that:

        \begin{equation}
        \begin{split}
            P(A) &= 0.08\\
            P(B|A) &= 0.98\\
            P(B|A') &= 0.03\\
        \end{split}
        \end{equation}

        where:
        \begin{equation}
        \begin{split}
            P(B) &= P(B|A) \cdot P(A) + P(B|A') \cdot P(A')\\
            &= (0.98 \cdot 0.08) + (0.03 \cdot 0.92)\\
            &= 0.106\\~\\
            P(A') &= 1 - P(A)\\
            &= 1 - 0.08\\
            &= 0.92
        \end{split}
        \end{equation}

        when solved is:
        \begin{equation}
        \begin{split}
            P(A|B) &= \frac{0.98 \cdot 0.08}{0.106}\\
            &= 0.7396\\
            & \approx \textbf{74\%}
        \end{split}            
        \end{equation}

\newpage
        \item The conditional probability that a patient is NOT actually infected given they test positive:\\

        Considering that:

        \begin{equation}
        \begin{split}
            P(A') &= 0.92 = 1 - 0.08\\
            P(B|A') &= 0.03\\
            P(B|A) &= 0.98\\
        \end{split}
        \end{equation}

        where:
        \begin{equation}
        \begin{split}
            P(B) &= P(B|A') \cdot P(A') + P(B|A) \cdot P(A)\\
            &= (0.03 \cdot 0.92) + (0.98 \cdot 0.08)\\
            &= 0.106\\~\\
            P(A) &= 1 - P(A')\\
            &= 1 - 0.92\\
            &= 0.08
        \end{split}
        \end{equation}

        when solved is:
        \begin{equation}
        \begin{split}
            P(A'|B) &= \frac{0.03 \cdot 0.92}{0.106}\\
            &= 0.2604\\
            & \approx \textbf{26\%}
        \end{split}            
        \end{equation}

        or more simply as:

        \begin{equation}
        \begin{split}
            P(A'|B) &= 1 - 0.7396\\
            &= 0.2604\\
            & \approx \textbf{26\%}
        \end{split}            
        \end{equation}\\

        \item The conditional probability that a patient is actually infected given they test negative:\\

        Considering that:

        \begin{equation}
        \begin{split}
            P(A) &= 0.08\\
            P(B'|A) &= 0.02 = 1 - 0.98\\
            % P(B'|A') &= 0.97 = 1 - 0.03\\
        \end{split}
        \end{equation}

        where:
        \begin{equation}
        \begin{split}
            P(B') &= 1 - P(B)\\
            &= 1 - 0.106\\
            &= 0.894
        \end{split}
        \end{equation}

        when solved is:
        \begin{equation}
        \begin{split}
            P(A|B') &= \frac{0.02 \cdot 0.08}{0.894}\\
            &= 0.0018\\
            & \approx \textbf{0.18\%}
        \end{split}            
        \end{equation}

        \item The conditional probability that a patient is NOT actually infected given they test negative:\\

        Considering that:

        \begin{equation}
        \begin{split}
            P(A') &= 0.92 = 1 - 0.08\\
            % P(B'|A) &= 0.02 = 1 - 0.98\\
            P(B'|A') &= 0.97 = 1 - 0.03\\
        \end{split}
        \end{equation}

        where:
        \begin{equation}
        \begin{split}
            P(B') &= 1 - P(B)\\
            &= 1 - 0.106\\
            &= 0.894
        \end{split}
        \end{equation}

        when solved is:
        \begin{equation}
        \begin{split}
            P(A'|B') &= \frac{0.97 \cdot 0.92}{0.894}\\
            &= 0.9982\\
            & \approx \textbf{99.82\%}
        \end{split}            
        \end{equation}
        
        or more simply as:

        \begin{equation}
        \begin{split}
            P(A'|B') &= 1 - 0.0018\\
            &= 0.9982\\
            & \approx \textbf{99.82\%}
        \end{split}            
        \end{equation}
    \end{enumerate}\\

    \item Suppose you have a biased coin with probability 0.2. Flip the coin until you get heads for the first time, record this number as $N$. Event $A$ when $N$ is at least 50. Event $B$ is when $N$ is at least 100.\\
    
    $P(B|A)$ can be expressed as:
        \begin{equation}
        \begin{split}
        P(B|A) &= P(A \cap B) / P(A)\\
        &= \sum_{n=100}^{\infty} (1 - 0.2)^{n-1} \cdot 0.2) / \sum_{n=1}^{50} (1 - 0.2)^{n-1} \cdot 0.2)
        \end{split}
        \end{equation}
        
    which can be simplified into a geometric series:
        \begin{equation}
        \begin{split}
            P(B|A) &= (1 - 0.2)^{99}) / (1 - 0.2)^{49})\\
        \end{split}
        \end{equation}

    \item Suppose you just turn onto a road, whose length is 10 miles. At each of the 10 mile-markers on the road, there is a 20\% probability of there being a police car trying to catch speeders, independently for each mile. You proceed as follows. Until you see your first police officer, you drive at 60 miles per hour. After you see your first police officer, you drive the rest of the distance at 50 miles per hour. Let $X$ be the random variable that denotes the number of miles you drive on this road until you see a police car; if you never see a police car on this road, then let $X = 10$ as the max road length is only 10 miles.
    
    \begin{enumerate}
        \item The amount of time it takes to reach the end of the road as a function of $X$ can be expressed as:

        \begin{equation}
            \begin{cases}
            X/60 + (10-X)/50 &\text{for}\ x \leq 10\\
            0 &\text{otherwise}
            \end{cases}
        \end{equation}\\
        
        % \begin{equation}
        %     \mathrm{E}[T] = X/60 \cdot (1 - (1 - 0.2)^{10}) + 10 / 50 \cdot (1 - (1 - 0.2)^{10})
        % \end{equation}\\

        \item The p.m.f. of $X$ for all possible values of $P(X = x)$ can be computed as:

        \begin{equation}
        \begin{split}
            P(X = 1) &= 0.2\\
            P(X = 2) &= 0.8 \cdot 0.2\\
            P(X = 3) &= 0.8^{2} \cdot 0.2\\
            P(X = 4) &= 0.8^{3} \cdot 0.2\\
            P(X = 5) &= 0.8^{4} \cdot 0.2\\
            P(X = 6) &= 0.8^{5} \cdot 0.2\\
            P(X = 7) &= 0.8^{6} \cdot 0.2\\
            P(X = 8) &= 0.8^{7} \cdot 0.2\\
            P(X = 9) &= 0.8^{8} \cdot 0.2\\
            P(X = 10) &= 0.8^{9} \cdot 0.2\\
        \end{split}
        \end{equation}

        \begin{equation}
            P(X = x) = \frac{1}{1-0.8^{x}} \cdot 0.8^{x-1} \cdot 0.2 \text{ for}\ x = 1,2,3,\hdots,10\\
        \end{equation}\\    

        \item The relationship between the distribution of $X$ and the geometric distribution is such that $X$ is a geometric r.v. where each mile marker is a Bernoulli trial (seeing a police officer), and $X$ is the number of miles until the first success within 10 miles.\\
    \end{enumerate}

    \item Suppose $C_1, C_2, C_3$ are three different biased coins, whose probability of heads equals 0.4, 0.5, and 0.2 respectively. Suppose the three coins look and feel identical, and are placed together in a box. You randomly picked a coin from the box and flipped it 10 times. Let $A$ denote the event you randomly chose coin $C_1$. Let $B$ denote the event that you got exactly 4 heads out of the 10 coin flips. 

    \begin{enumerate}
        \item $P(A \cap B)$ can be computed as:
            \begin{equation}
            \begin{split}
                P(A \cap B) &= P(B|A) \cdot P(A)\\
                &= 0.2508 \cdot 0.3333\\
                &= 0.0836
            \end{split}
            \end{equation}

            where:
            \begin{equation}
            \begin{split}
                P(B|A) &= \frac{10!}{(10-4)! \text{\space} 4!} \cdot 0.4^{4} \cdot 0.6^{6}\\
                &= {10\choose 4}\cdot 0.4^{4} \cdot 0.6^{6}\\
                &= 0.2508\\~\\
                P(A) &= \frac{1}{3}\\
                &= 0.3333
            \end{split}
            \end{equation}
            
        \item $P(B)$ can be computed as:
            \begin{equation}
            \begin{split}
                P(B) &= P(B|A) \cdot P(A) + P(B|A') \cdot P(A')\\
                &= (0.2508 \cdot 0.3333) + (0.0977 \cdot 0.6667)\\
                &= 0.1487
            \end{split}
            \end{equation}

            where:
            \begin{equation}
            \begin{split}
                P(B|A') &= \biggl({10\choose 4} \cdot 0.5^{4} \cdot 0.5^{6} \cdot 0.33 \biggl) + \biggl({10\choose 4} \cdot 0.2^{4} \cdot 0.8^{6} \cdot 0.33\biggl)\\
                &= 0.0977\\~\\
                P(A') &= 1 - P(A)\\
                &= 0.6667
            \end{split}
            \end{equation}

        \item $P(A|B)$ can be computed as:
            \begin{equation}
            \begin{split}
                P(A|B) &= P(A \cap B) / P(B)\\
                &= 0.0836 / 0.1487\\
                &= 0.5622
            \end{split}
            \end{equation}\\
    \end{enumerate}
    
    \item Suppose $Z$ is a r.v. such that:
    \begin{equation} \nonumber
        p_Z(x) = 
            \begin{cases}
                0.5 & \text{if } x = -3.1\\
                0.1 & \text{if } x = -2\\
                0.1 & \text{if } x = 6\\
                0.3 & \text{if } x = 12\\
                0 & \text{otherwise}\\
            \end{cases}
    \end{equation}\\

    and $Y$ is another r.v. such that $Y = Z^{2} - Z + 3$.\\

    \begin{enumerate}

        \item $\mathbb{E}{[Y]}$ can be computed as:

        \begin{equation}
        \begin{split}
            \mathbb{E}{[Y]} &= \mathbb{E}{[Z^2 - Z + 3]}\\
            &= \mathbb{E}{[Z^2]} - \mathbb{E}{[Z]} + 3\\
            &= 52.005 - 2.45 + 3\\
            &= 52.555
        \end{split}
        \end{equation}

        where:
        \begin{equation}
        \begin{split}
            \mathbb{E}{[Z^{2}]} &= (-3.1^{2}  \cdot 0.5) + (-2^{2} \cdot 0.1) + (6^{2} \cdot 0.1) + (12^{2} \cdot 0.3)\\
                &=4.805 + 0.400 + 3.600 + 43.200\\
                &=52.005\\~\\
            \mathbb{E}{[Z]} &= (-3.1  \cdot 0.5) + (-2 \cdot 0.1) + (6 \cdot 0.1) + (12 \cdot 0.3)\\
                &=-1.55 + -0.20 + 0.60 + 3.6\\
                &=2.45
        \end{split}
        \end{equation}
        
        \item Suppose $W$ is a r.v. such that:
        
            \begin{equation} \nonumber
                p_W(x) = 
                    \begin{cases}
                        0.5 & \text{if } x = -3.1\\
                        0.1 & \text{if } x = -2\\
                        0.1 & \text{if } x = 6\\
                        0.3 & \text{if } x = 12\\
                        0 & \text{otherwise}\\
                    \end{cases}
                \end{equation}\\
        
            $Var[W]$ can be computed as:

            \begin{equation}
            \begin{split}
                Var[W] &= \mathbb{E}{[(W - \mathbb{E}{[W]})^2]}\\
                &= \mathbb{E}{[W^2]} - (\mathbb{E}{[W])^2}\\
                &= 52.005 - 6.0025\\
                &= 46.0025
            \end{split}
            \end{equation}

            where:
            \begin{equation}
            \begin{split}
                \mathbb{E}{[W^{2}]} &= (-3.1^{2}  \cdot 0.5) + (-2^{2} \cdot 0.1) + (6^{2} \cdot 0.1) + (12^{2} \cdot 0.3)\\
                    &=4.805 + 0.400 + 3.600 + 43.200\\
                    &=52.005\\~\\
                (\mathbb{E}{[W])^{2}}&= ((-3.1  \cdot 0.5) + (-2 \cdot 0.1) + (6 \cdot 0.1) + (12 \cdot 0.3))^{2}\\
                &= (-1.55 + -0.02 + 0.60 + 3.6)^{2}\\
                &= 2.45^{2}\\
                &=6.0025
            \end{split}
            \end{equation}
    \end{enumerate}
\end{enumerate}
\end{document}